\chapter{Related Work}

\section{3D Model Retrieval}

An important part of 3D model retrieval are the feature extraction algorithms. These have been, and still are, an active research topic and thus there are many algorithms available. R. Osada et al. describe the D2 shape distribution \cite{d2_shape_distribution} which results in a histogram of distances between randomly sampled pairs on the 3D model's surface. Y.-J. Liu et al. then build upon this and propose ClusterD2+Color and ClusterAngle+Color \cite{cluster_d2_color}, which work in a similar way, but also takes the color and texture of the 3D model into consideration. Kazhdan et al. describe a rotation invariant spherical harmonics representation of 3D models \cite{spherical_harmonics} making use of voxel grids and spherical harmonics functions to compute a descriptor. Yet another method, Spin Images \cite{spin_descriptor}, performs well at matching partial surfaces of 3D models.\\
However, these algorithms are only useful as part of a retrieval system that can use them to search through a database. Cineast was initially conceived by L. Rossetto \cite{cineast_rossetto} as a retrieval system for videos. R. Gasser then extends Cineast to a multimedia retrieval system \cite{cineast_gasser} supporting various other types of media, such as 3D models. Flickner et al. \cite{qbic_system} first described the QBIC System, where the user can specify or sketch an example image that the retrieval system then uses to find similar looking content. The retrieval system user interfaces demonstrated in \cite{cineast_gasser, 3d_model_retrieval_vranic, cluster_d2_color} all make the use of such a system.

\section{Virtual Reality Sculpting}

While virtual reality sculpting has not yet attracted the attention of the scientific research community, it has gained a lot of popularity in the games industry in recent years.
Medium\footnote{\url{https://www.oculus.com/medium}} offers a vast and mature toolbox to enable users to sculpt professional grade character models using the Oculus VR hardware.
Tilt Brush\footnote{\url{https://www.tiltbrush.com/}} takes a more painterly approach and mostly uses flat brush strokes to create scenes with a stylized hand drawn look. However, it also has some brush strokes that are 3D or can be animated. Quill\footnote{\url{https://quill.fb.com/}} uses a similar style and also offers extensive tools to let the user animate scenes smoothly. SculptVR\footnote{\url{http://www.sculptrvr.com/}} takes a voxel based approach similar to some of the methods discussed in this thesis.
