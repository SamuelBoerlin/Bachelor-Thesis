\chapter{Introduction}

Internet search engines like Google are part of our daily lives, so much so that the verb "to google" was added to the Oxford English Dictionary in 2006 and has been used commonly ever since. However, in the recent years interest in 
both on- and offline search engines for media other than text documents has increased a lot, and one such type of media are 3D models. Looking for a specific 3D model in a large collection is a laborious task that can be solved by multimedia retrieval applications capable of searching for 3D models not only by labels or title, but also by shape and looks. In this bachelor's thesis we present a novel way of searching databases for certain 3D models by making use of virtual reality to enable the user to sculpt 3D sketches that the Cineast multimedia retrieval system can then use as input.

\section{Motivation}

Multimedia retrieval applications presented in recent research use a drawing area where the user can draw a 2D sketch of the model they would like the search. This may work well for simple shapes but it does not allow the user to make full use of the 3D space. Furthermore, sculpting virtual models with keyboard and mouse is not intuitive due to the mismatch between the 3D nature of sculpting and the 2D movement of the mouse. Therefore, our goal is to build a novel system in virtual reality making use of its 3D motion tracking controllers that enable the user to sculpt 3D shapes or figures and then use them as a sketch to automatically search a database for similar 3D models.

\section{Approach}
\label{sec:approach}

During the semester of Fall 2019 C. Tejera, K. Nemmour and S. Börlin built a prototype of such an application in the Unity game engine for the Modern Human-Machine Interaction seminar at the University of Basel.
We build upon this prototype by extending its functionality, improving its user interfaces and implementing a new feature module in Cineast in order to increase the accuracy of the database search for colored or textured 3D models.